\documentclass[11pt,a4paper]{report}


\usepackage{listings}
\usepackage{inconsolata}

\usepackage[margin=1in]{geometry}

\usepackage{enumitem}






%% This allows us to use lstlistings inside list-descriptions:
\makeatletter
\let\orig@item\item

\def\item{%
    \@ifnextchar{[}%
        {\lstinline@item}%
        {\orig@item}%
}

\begingroup
\catcode`\]=\active
\gdef\lstinline@item[{%
    \setbox0\hbox\bgroup
        \catcode`\]=\active
        \let]\lstinline@item@end
}
\endgroup
\def\lstinline@item@end{%
    \egroup
    \orig@item[\usebox0]%
}
\makeatother

%% ============================================================






\lstset{basicstyle=\ttfamily}




\newenvironment{codedescription}{%
   \setdescription{leftmargin=\parindent,labelindent=\parindent}
   \renewcommand\descriptionlabel[1]{\hspace{\labelsep}##1 --}
   \begin{description}%
}{%
   \end{description}%
}






\begin{document}

\chapter{Notes on LEMS Lite syntax}

\section{ComponentArrays}

\section{Connectors}


\subsubsection{Builtin Connector Types}

\begin{codedescription}
    \item[\lstinline|<AllToAll|] sdf
    \item[\lstinline|<AllToAllConnector allow_self_connections="false">|] jkl
    \item[\lstinline|<OneToOneConnector|]  jkl
    \item[\lstinline|<FixedProbabilityConnector p_connect, allow_self_connections=True>|] jkl
    \item[\lstinline|<FixedNumberPreConnector n, allow_self_connections=True,>|] jl
    \item[\lstinline|<FixedNumberPostConnector n, allow_self_connections=True,>|] jkl
    \item[\lstinline|<DistanceDependentProbabilityConnector>|] jkl
    \item[\lstinline|<FromListConnector>|] jkl
    \item[\lstinline|<FromFileConnector>|] jkl
    \item[\lstinline|<CSAConnector>|] jkl
\end{codedescription}



%<connect dst_param="weight" value="from_file:my_weights.txt"/>
%<connect dst_param="weight" value="from_list:my_weights.txt"/>
%<connect dst_param="weight" value="from_expression:my_weights.txt"/>
%<connect dst_param="weight" value="fixed:1.0"/>
%
%
%
%
%<connections>
%  <PyNN:AllToAllConnector allow_self_connections>
%  <PyNN:OneToOneConnector>
%  <PyNN:FixedProbabilityConnector p_connect, allow_self_connections=True>
%  <PyNN:FixedNumberPreConnector n, allow_self_connections=True,>
%  <PyNN:FixedNumberPostConnector n, allow_self_connections=True,>
%  <PyNN:DistanceDependentProbabilityConnector>
%  <PyNN:FromListConnector>
%  <PyNN:FromFileConnector>
%  <PyNN:CSAConnector>
%
%  <PairwiseBooleanExpression >
%  <PairwiseProbabilisticExpression>
%  
%  <StevesMondoConnector>
%</connections>




\section{Tags for specifying Parameter values}


%<from_expression sample="0.5" />
%<from_expression sample="sin((i,j)) + uniform(a,b)" />
%<from_list values="[]" />
%<from_file format='flat' filename="" />


  <PairwiseBooleanExpression if='i<j' index_order=>
  <PairwiseProbabilisticExpression p=' '>

  <StevesMondoConnector>



\section{Mathematical Expressions}



\subsubsection{Operators}

What to do about ternary operator?



\subsubsection{Standard Library functions}
\lstset{language=C}

\begin{codedescription}
\item[\lstinline|clip(x,x_min,x_max)|] asdf
\item[\lstinline|clip_max(x,x_max)|] asdf
\item[\lstinline|clip_min(x,x_min)|]  asdf
\vspace{0.25cm}
\item[\lstinline|abs(x)|]  ....
\item[\lstinline|floor(x)|] ...
\item[\lstinline|ceil(x)|]  ...
\item[\lstinline|round(x)|]  ...
\vspace{0.25cm}
\item[\lstinline|sqrt(x)|]  ...
\item[\lstinline|power(x, y)|]  ...
\item[\lstinline|exp(x)|]  ...
\item[\lstinline|log(x)|]  (log10)
\item[\lstinline|ln(x)|]  ...
\vspace{0.25cm}
\item[\lstinline|sin(x)|]  ... (in rads )
\item[\lstinline|cos(x)|]  ... (in rads )
\item[\lstinline|tan(x)|]  ... (in rads )
\end{codedescription}


Don't expose 'pi' and 'e', because they can be written as constants





\subsubsection{Random Numbers}

\lstset{language=C}
\begin{codedescription}
\item[\lstinline|uniform(a,b)|] Uniform numbers between a,b
\item[\lstinline|normal(mean,stddev)|] 
\end{codedescription}




\end{document}
